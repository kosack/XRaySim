\documentclass{article}
\begin{document}

	\pagestyle{plain}

	\title{XRaySim Documentation}
	\author{Karl Kosack (kosack@andrew.cmu.edu)}
	\maketitle

	\tableofcontents
	\newpage

\section{Introduction}

XRaySim is a computer program which simulates x-ray diffraction
through a polycrystalline material. In particular, it is intended as a
simulation of a 3D x-ray microscope.  In such a setup, an x-ray beam
is passed through a crystal sample which is made up of various crystal
grains of differing orientation. As the beam travels through the
sample, it is diffracted by the grains, producing a pattern on a 2-d
detector.  The pattern produced when the sample is rotated in the beam
can be used to calculate the orientation of an individual grain in the
sample. 

\section{Technical Info}

\subsection{Overview}

In this simulation, a sample is represented as a set of cubic {\em
voxels}, or volumetric pixels, which contain information about grain
position and orientation. Using a ray-tracing technique, the path of
an x-ray beam through the sample is calculated and diffracted beams
are sent out when necessary. The size and 3D resolution of the sample
is only limited by memory size and computation time.  Due to the large
storage and computation needs of voxel-based visualization (i.e. a
256x256x256 set of voxels takes around 32Mb of memory assuming each
only holds 1 character worth of information), it should be noted that
more complex samples will require substantial free memory.

\subsection{How it works}

XRaySim is a specialized ray tracer which takes into account simply
x-ray optics.  XRaySim first loads a list of crystal grains from a
datafile and computes the scattering vectors for each based on their
structure and orientation. Information describing the sample volume is
then loaded into memory as a very large array of integers, each an
index to a grain number.  When a simulated ray is fired from the
emitter through the sample, the entry point into the sample is
calculated and all voxels along the path of the ray are tested for
intersection.  If intersection occurs, the orientation of the grain
residing at the intersection point is tested against the condition for
x-ray scattering.  If scattering occurs, a ray is fired from the
intersected voxel in the correct direction.  If at any time a
scattered ray intersects the detector, the point that was hit is
recorded. The sample itself sits on a simulated translation/rotation
stage so that it can be placed in any arbitrary position and
orientation.

%%%%%%%%%%%%%%%%%%%%%%%%%%%%%%%%%%%%%%%%%%%%%%%%%%%%%%%%%%%%%%%%%
\section{Usage}

%================================================================
\subsection{Starting from the Command Line}

Start XRaySim by typing ``xraysim'' at the console command line.
XRaySim accepts several command-line arguments which are as follows:

	\begin{quote}
		\begin{description}
		\item [-t] Text mode - no graphics display
		\item [-d] Debug mode - output debugging info (highly verbose)
		\item [-h] Display help info
		\end{description}
	\end{quote}

%================================================================
\subsection{Graphical Interface}

% ADD MORE HERE!

XRaySim provides a relatively easy to use GUI interface to controlling
the simulation.  When the interface starts, you are presented with
two windows -- the experiment window, and the detector window.  


\subsubsection{Experiment Window}

The Experiment Window allows the user to manipulate the various
aspects of the simulator and run some preset scans.  The window is
divided into two panels: the experiment control panel and the display
control panel.  The former controls aspects of the experiment, and the
latter controls the display of the experiment and detector windows.
The experiment window also shows a 3D representation of the experiment
setup.  The various controls in the ``Display Controls'' panel change
how this display works.

	\begin{description}

		\item[Omega, Chi, Phi] For each Euler angle, you can specify a
		starting position (changing which will immediately rotate the
		sample to the specified position), a step, and a stopping
		position, all in radians.  The step and stop fields are used
		by the {\em AutoRotate } scan, mentioned later. The three
		check boxes next to these fields are for activating or
		deactivating the ``motors'' which move that particular
		rotation.  They have no effect unless a scan is running. 

		\item[tx, ty, tz] These fields control the translation of the
		sample in the x, y, and z directions. 

		\item[Beam] When this button is active, the beam and
		diffracted beams can be seen in the experiment display.  It is
		useful for manual manipulation of the simulation.  You do not
		need to turn the beam on to do an AutoRotate scan, but if you
		do, you will be able to see the beam during the scan. 

		\item[AutoRotate] Press this button to begin an AutoRotate
		scan.  The sample will start at the current rotation and
		rotate in the beam along each activated Euler angle until the stopping
		angle is hit.

		\item[Fire One] Press this button to fire one ray at the
		sample.

		\item[Boundary Search] Press this button to begin a grain
		boundary search. The grain boundary search will automatically
		map the boundary of the grain that the beam is currently in.
		To perform this search, you must first rotate the sample so
		that the beam is causing a Bragg peak to be visible on the
		detector.  Then, you must select the peak to watch on the
		detector window by drawing a box around it.  Finally, push the
		Boundary Search button and the scan will begin from the
		current starting position.

		\item[Map crystal] Runs an automatic scan to map the central 
		grain and its neighbors.

		\item[Beam Divergence, Detector Distance] These
		parameters control the setup of the experiment.  Click them to
		change.

		\item[Sampling vol min/max] These fields control the size of
		the sampling volume -- the area inside the sample from which
		x-rays are to be diffracted.  Min and max are measured from the
		center of the sample when it is not translated, so if you want
		to restrict diffraction to the central grain, make sure these
		boundaries fall within that grain. In reality, such a volume
		may be restricted by a series of conical slits placed in
		front of the detector which only allow rays originating from a
		particular position to pass through.
		

		\item[Output to:] Choose either ``Display'' or ``Logfile''.
		For instance, during an AutoRotate scan, if ``Display'' is
		chosen, the detector window will update to show the position
		of rays that hit it when the scan is done. If ``log file'' is
		chosen, detailed information about the position and intensity
		versus the rotation angle and translation will be written to
		``scope.log'' so that another program can read and analyze it.

	\end{description}

\bigskip

	\begin{description}

		\item[Display Update] Choose this to continuously update the
		experiment window during a scan (useful for watching the
		progress of the scan, but it slows things down a bit). 

		\item[Detector Update] Works the same as the Display Update
		toggle, but for the detector window.  This tends to slow down
		the simulation considerably and isn't recommended.

		\item[Sample Reference Frame] Turn this on to place the
		experiment display in the fixed reference frame of the sample.

		\item[r,theta,psi] Turn these knobs to move the camera
		position for the experiment window.

		\item[Voxel Display] Choosing a value other than ``None'' will
		cause the voxels that make up the sample to be displayed in a
		color-coded fashion.  Only enable this if there are under
		10000 voxels (otherwise, it is very slow).

	\end{description}




\subsubsection{Detector Window}
This window shows a graphical representation of the state of the
detector.  Brighter spots mean more rays have hit that position. If an
AutoRotate scan is running with the log file option turned on, the
detector is cleared between each rotation, so it cannot be displayed.
If you want to see the detector displayed, turn the log off.  You can
select a spot on the detector by clicking and dragging across the
detector window surface.  To clear this window, choose
Detector---Reset from the menu bar in the experiment window.  You can
also save the current state of the detector from the menubar.



%%%%%%%%%%%%%%%%%%%%%%%%%%%%%%%%%%%%%%%%%%%%%%%%%%%%%%%%%%%%%%%%%
\section{Programming Info}

%================================================================
\subsection{Data Structures}

\subsubsection{XRayScope}

The {\em XRayScope} object is the ``core'' of the program and is
really all that is needed to perform all functions of the simulator.
Any sort of user interface that is used must create a new XRayScope
and call its functions, which include:

	\begin{description}
	
	\item [loadSample()] load grain information into the simulation
	and create a new set of voxels for the sample
	\item [setDetectorDist(dist)]  set the distance between the detector
	and the sample
	\item [getDetectorDist()] returns current detector distance
	\item [setDetectorSize(width,height)] set the physical width and height of
	the detector
	\item [setDetectorRes(xres,yres)] set the horizontal and vertical
	resolution of the detector in pixels
	\item [setProbeVolumeSize(min, max)] restrict the part of the
	beam which interacts with the sample. Min and max are relative to
	the center of the sample.
	\item [getProbeVolumeMax() and getProbeVolumeMin()] 
	return the current probe volume max/min value which was set in 
	setProbeVolumeSize()
	\item [setBeamDiv(div)] set the divergence of the x-ray beam
	\item [getBeamDiv()] returns the current beam divergence
	\item [rotateSample(phi,theta,psi)] rotate sample to the specified
	Euler angles (in radians)
	\item [translateSample(tx,ty,tz)] translate the sample
	\item [activate()] activate the x-ray beam and fire one ray at the
	sample. 
	\item [setLogFilename(filename)] set the name of the file where
	output should be written (default is 'scope.log').
	\item [writeLogHeader()] Output the current state of the machine
	to the logfile
	\item [writeLogEntry()] write a log entry to the logfile 
	\item [watchDetectorSpot(minx,miny,maxx,maxy)] Returns the
	integrated intensity of an area of the detector.

	\end{description}


For instance, if you wanted to simulate rotating the sample while
it is in the beam, you would do the following:

\begin{verbatim}
    
    XRayScope scope;
    double i;

    // Set the defaults
    scope.loadSample("sample.xrs");
    scope.setDetectorDist(5);
    scope.setDetectorSize(10,10);
    scope.setDetectorRes(500,500);
    
    // Rotate the sample and fire the beam
    for(i=0 ; i<2*PI; i+=0.001) {
            
        scope.rotateSample(i,0,0);
        scope.activate();

    }

    // Output the final state of the detector
    scope.writeLogHeader();					
    scope.writeLogEntry();


\end{verbatim}

Note that {\em writeLogHeader()} and {\em writeLogEntry()} are
separate functions---this allows for a more flexible data format.
Whenever {\em writeLogHeader()} is called, a block of text is written to the
current log file which describes all of the global parameters that are
in use (e.g. wavelength and detector size).  You only need to write
this data out once unless you change one of these global parameters
during a scan.  Whenever {\em writeLogEntry()} dumps a series of lines to
the current log file, which describe the position of each detector
spot and the current sample orientation and translation. In the
example above, the integrated detector information is written, since
the only one log entry is written after the scan has completed.
Alternatively, one could call {\em writelogHeader()} at the beginning of the
scan, and at each step, clear the detector image, fire the beam, and
call {\em writeLogEntry()} so that detector information will be saved at
each angle (the default AutoRotate scan works this way, so see the
code in scans.h for an example).

\subsubsection{Grain}
A {\em Grain} object stores information about a particular crystal
grain, including its orientation, its structure, and a list of
scattering vectors which are precomputed when the Grain is
created. The method {\em Grain::getDiffraction()} is where the mathematics
of crystal diffraction is actually carried out. Grains should only be
created in the {\em XRayScope::loadSample()} method, so there is really no
need to go into detail about their use here.  For more information,
look in the xraysim.h file.


\subsubsection{Emitter}
The {\em Emitter} object is a simple data structure which holds the
current position of the x-ray emitter.  An emitter is created when a
new XRayScope is made, and its properties are automatically set. 


\subsubsection{Detector}
The {\em Detector} object simulates a two-dimensional photon
detector. A new Detector is automatically created when an XRayScope is made.  
The detector records the position of intersected rays in an image
buffer.  All attributes of the detector should be set through the
corresponding XRayScope methods (see above). 

\subsubsection{Volume}
The {\em Volume} object is simply a container structure for a set of
voxels (defined by the {\em Voxel} object).  In short, a Volume
represents the entire crystal sample.  

\subsubsection{Voxel}
The {\em Voxel} object is the storage class for a volumetric pixel
(three dimensional).  Voxels must be orthogonal to the ``world''
coordinate axes to speed up intersection testing, so voxels are always
fixed in space.  Because of this, rotating the sample actually
involves keeping the sample fixed and rotating the detector and
emitter about it (although that should be transparent to the user of
the program, since it is taken care of by the XRayScope object).  Each
voxel contains a grain index which is the index of the crystal grain
to which it belongs.

\subsubsection{Ray, Isect, etc.}
The {\em Ray, Isect}, and related objects are data types used
throughout the simulation.  A Ray contains information about a single
ray of light, including its wavelength, intensity, direction, and
position. The Isect object represents the intersection between a Ray
and something in the scene (i.e. the detector or sample). 


%================================================================
\subsection{Mathematics}

Most of the mathematics involved in the simulation that relate to
crystallography can be found in the {\em Grain::getDiffraction()}
method and in the Grain object constructor, {\em Grain::Grain()}. It
should be noted that all of the mathematics in this simulation assume
that the crystal has a {\bf cubic bravais lattice} with a structure of
FCC, BCC, or SC.

\subsubsection{Scattering vectors}
In the Grain constructor, a list of scattering vectors ($\vec G_i$) is computed
using the structure of the crystal and its orientation.  This is done
by iterating over all acceptable miller indices (following the
selection rule corresponding to the grain's structure) and using
Equation \ref{G}.

	\begin{equation} \label{G}
	\vec{G} = h \vec{B_x} + k \vec{B_y} + l \vec{B_z}
	\end{equation}

	\begin{center}
	$\vec{B_i} = M (\frac{2\pi}{a}) \hat i	$ \\
	\end{center}

Where $M$ is the rotation matrix which describes the orientation of
the grain, (h,k,l) are Miller indices, $a$ is the lattice constant and
($\vec B_i$) are the basis vectors.

\subsubsection{Diffraction}
The {\em getDiffraction()} method returns a list of diffracted rays given an
incident ray (wave vector $\vec k_i$) and that ray's intersection point.
The condition for diffraction is shown in Equation \ref{DIFFCOND}.

	\begin{equation} \label{DIFFCOND}
	G^2 = -2 \vec{k_i} \cdot \vec{G} 
	\end{equation} 

Where, $G$ is a scattering vector (a list of which is precomputed in
Grain::Grain). This is implemented by iterating over scattering
vectors and testing whether $ \vert G^2+2*\vec k_i \cdot \vec G \vert $ is less
than a tolerance (referred to in the program as the {\em beam
divergence} because a larger value simulates a wider beam).  The
diffracted beam ($\vec k_f$) is then found by the formula in Equation \ref{KF}.

	\begin{equation} \label{KF}
	\vec{k_f} = \vec{k_i} + \vec{G}		
	\end{equation}

The closer the beam divergence is to 0, the finer the beam will be,
and the less diffraction will occur.  For instance, to make a
simulated Laue transmission image, set the beam divergence to
something high (like 1.0) by calling the {\em XRayScope::setBeamDiv()}
method and activate the XRayScope.


%================================================================
\subsection{Data Input/Output}

\subsubsection{Input}
XraySim reads sample information in from a sample datafile.  The
format of the file is as follows:

\begin{verbatim}
    
    XRAYSIM-SAMPLE 0.99

    # This is the default datafile, sample.xrs, which is loaded when XRaySim starts
    # Use it as a template for other sample files

    # KEYS/VALUES
    > width=0.05 
    > height=0.05 
    > depth=0.05 
    > voxsize=0.0025
    > lambda=1.5405e-8

    # GRAIN LIST BEGINS HERE:
    #------------------------------------------------------------------- 
    #omega  chi phi a       graintype (FCC, BCC, or SC)
    #------------------------------------------------------------------- 

    0   0   0   5.6402e-8   FCC
    25  16  2   5.6402e-8   FCC
    13  24  15  5.6402e-8   FCC
    45  0   0   5.6402e-8   FCC
    45  45  0   5.6402e-8   FCC
    45  45  45  5.6402e-8   FCC
    45  45  45  5.6402e-8   FCC
    45  28  130 5.6402e-8   FCC
    0   15  274 5.6402e-8   FCC
    0   0   12  5.6402e-8   FCC
    15  0   5   5.6402e-8   FCC
    15  15  300 5.6402e-8   FCC
    300 1   99  5.6402e-8   FCC

\end{verbatim}


The first line is just a file header, and can be ignored.  Any line
that begins with a '\#' symbol is a comment, and will be ignored when
the program loads the data. Lines which begin with a '$>$' symbol
must contain a parameter and a value separated by an '$=$' sign. All
other lines contain a list of grains which are in the sample.  The
columns are described in the figure above.  If you don't specify all of
the parameters listed above, the default values will be used (which
may cause unexpected results).

\subsubsection{Output}

All XRaySim data is outputted in the following format:

	\begin{list}{-}{}
		\item Lines beginning with a \# are comments
		\item Lines beginning with $>$ contain a parameter key/value
		pair separated by an $=$ sign. (eg: {\em $>$ lambda=1.54e-6})
		\item All other lines are tab-delimited data with the
		following columns: ($\omega$  $\chi$  $\phi$  tx  ty  tz dety
		detz intensity).  The first three are the sample rotation, the
		second three are the translation, the next two are the
		detector position, and the last is the number of rays which
		hit that position
	\end{list}


\subsection{Writing Scans}

	XRaySim has several scans already programmed into it, including
	{\em AutoRotate} which outputs detector information as the sample
	rotates about a fixed axis, and several scans for mapping the
	grain microstructure which assume a probe volume which is smaller
	than the grain size.  However, it is fairly easy to add new scans
	(though you need to recompile the program each time, since there
	is no macro language or plug-in support).  All of the scans are
	defined in the files {\em scans.c} and {\em scans.h}, and they are
	simply functions which call XRayScope methods.  Each function
	assumes that there is a global XRayScope object called {\em
	scope}, which is constructed in {\em main()}.
	
	To add a new scan, simply write a function that does what you
	want, and then call it either from {\em main()} if you want it to
	run in text mode, or from the graphical interface if you are more
	adventurous. The code to draw the graphical interface is in {\em
	interface.c}. See the FLTk manuals for info on programming it.
	You will notice that the preset scans also contain some code which
	is specific to the graphics interface (for button/widget callbacks
	and OpenGL display).  These are completely optional, and needn't
	be included in any new scans (unless you want watch an animated
	view of the scan).

	There is a MS Visual C++ project file which should work for
	recompiling XRaySim.  Note that the OpenGL and FLTk libraries must
	be installed.  It is possible to compile XRaySim without OpenGL if
	needed: remove the OpenGL libraries from the project (by editing
	the linker settings) and then comment out all lines which say
	``\#define USE\_OPENGL''.  All OpenGL code in XRaySim is enclosed by
	{\em \#ifdef}'s which should then instruct the compiler to ignore
	OpenGL calls. I've never tested this, so it may not work---you may
	need to edit the header files.

\end{document}






